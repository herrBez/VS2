\exercise{Esercizio 1}
{Let $(\alpha, C, A, \gamma)$ be a Galois connection. Prove that:
	\begin{enumerate}[label=(\Alph*)]
		\item $\gamma$ is injective, $\iff$
		\item $\alpha \circ \gamma = id$ $\iff$
		\item $\alpha$ is surjective.
	\end{enumerate} 
}
{
	Alcune definizioni utili:
	\begin{mydef}(Galois Connection)
		$(\alpha, C, A, \gamma)$ è una Galois connection se:
		\begin{enumerate}
			\item A, C sono poset
			\item $\alpha: C \longrightarrow A$ monotona
			\item $\gamma: A \longrightarrow C$ monotona
			\item $\forall c \in C.\ c \leq_C \gamma(\alpha(c))$
			\item $\forall a \in A.\ \alpha(\gamma(a)) \leq_A a$
		\end{enumerate}
	\end{mydef}
	\begin{mydef}(Funzione iniettiva)
		Una funzione $f: X \longrightarrow Y$ è iniettiva se:
		$$
		\forall a, b \in X .\  f(a) = f(b) \implies a = b
		$$
	\end{mydef}
	$\boxed{(B) \implies (A)}$
	Assumo che valga l'ipotesi (B) ovvero che $\alpha \circ \gamma = id$.
	Assumo per assurdo che non valga l'ipotesi (A) ovvero che $\gamma$ non sia iniettiva, quindi:
	$$
	\exists a, b \in A .\ \gamma(a) = \gamma(b) \land a \neq b
	$$
	Siano $a,b$ due elementi diversi di $A$ ($a \neq b$) che vengono mappati allo stesso elemento di $C$, ovvero $\gamma(a) = \gamma(b)$.
	Essendo $(C, \leq_C)$ un poset, per la riflessività della relazione $\leq_C$, 
	valgono anche le relazioni:
	\begin{enumerate}[label=\Roman*)]
		\item $\gamma(a) \leq_C \gamma(b)$ 
		\label{item:gamma(a):leq:gamma(b)}
		\item $\gamma(b) \leq_C \gamma(a)$
		\label{item:gamma(b):leq:gamma(a)}
	\end{enumerate}
	
	Partendo dalla relazione \ref{item:gamma(a):leq:gamma(b)} ottengo:
	\begin{displaymath}
	\begin{array}[\textwidth]{llr}
	\gamma(a) \leq_C \gamma(b) 					& \implies & \text{per la monotonia di $\alpha$} \\
	\alpha(\gamma(a)) \leq_A \alpha(\gamma(b))  & \implies & \text{per l'ipotesi (B)} \\
	a \leq_A b									& \\
	\end{array}
	\end{displaymath}
	
	Partendo dalla relazione \ref{item:gamma(b):leq:gamma(a)} ottengo:
	\begin{displaymath}
	\begin{array}[\textwidth]{llr}
	\gamma(b) \leq_C \gamma(a) 					& \implies & \text{per la monotonia di $\alpha$} \\
	\alpha(\gamma(b)) \leq_A \alpha(\gamma(a))  & \implies & \text{per l'ipotesi (B)} \\
	b \leq_A a									& \\
	\end{array}
	\end{displaymath}
	Siccome valgono al contempo $a \leq_A b$ e $b \leq_A a$ allora per l'antisimmetria del poset $(A, \leq_A)$
	$a = b$. Che è in contrasto con l'ipotesi iniziale che $a \neq b$.
	
	

}
\exercise{Esercizio 1}
{Let $(\alpha, C, A, \gamma)$ be a Galois connection. Prove that:
	\begin{enumerate}[label=(\Alph*)]
		\item $\gamma$ is injective, $\iff$
		\item $\alpha \circ \gamma = id$ $\iff$
		\item $\alpha$ is surjective.
	\end{enumerate} 
}
{
	Alcune definizioni utili:
	\begin{mydef}(Galois Connection)
		$(\alpha, C, A, \gamma)$ è una Galois connection se:
		\begin{enumerate}
			\item A, C sono poset
			\item $\alpha: C \to A$ monotona
			\item $\gamma: A \to C$ monotona
			\item $\forall c \in C.\ c \leq_C \gamma(\alpha(c))$
			\item $\forall a \in A.\ \alpha(\gamma(a)) \leq_A a$
		\end{enumerate}
	\end{mydef}
	\begin{mydef}(Funzione iniettiva)
		Una funzione $f: X \to Y$ è iniettiva se:
		$$
		\forall a, b \in X .\  f(a) = f(b) \implies a = b
		$$
	\end{mydef}
	
	
	
	%% INIZIO DIMOSTRAZIONE (B) => (A)%
	$\boxed{(B) \implies (A)}$
	Assumo che valga l'ipotesi (B) ovvero che $\alpha \circ \gamma = id$.
	Assumo per assurdo che non valga l'ipotesi (A) ovvero che $\gamma$ non sia iniettiva, quindi:
	$$
	\exists a, b \in A .\ \gamma(a) = \gamma(b) \land a \neq b
	$$
	Siano $a,b$ due elementi diversi di $A$ ($a \neq b$) che vengono mappati allo stesso elemento di $C$, ovvero $\gamma(a) = \gamma(b)$.
	Essendo $(C, \leq_C)$ un poset, per la riflessività della relazione $\leq_C$, 
	valgono anche le relazioni:
	\begin{enumerate}[label=\Roman*)]
		\item $\gamma(a) \leq_C \gamma(b)$ 
		\label{item:gamma(a):leq:gamma(b)}
		\item $\gamma(b) \leq_C \gamma(a)$
		\label{item:gamma(b):leq:gamma(a)}
	\end{enumerate}
	
	Partendo dalla relazione \ref{item:gamma(a):leq:gamma(b)} ottengo:
	\begin{displaymath}
	\begin{array}[\textwidth]{llr}
	\gamma(a) \leq_C \gamma(b) 					& \implies & \text{per la monotonia di $\alpha$} \\
	\alpha(\gamma(a)) \leq_A \alpha(\gamma(b))  & \implies & \text{per l'ipotesi (B)} \\
	a \leq_A b									& \\
	\end{array}
	\end{displaymath}
	
	Partendo dalla relazione \ref{item:gamma(b):leq:gamma(a)} ottengo:
	\begin{displaymath}
	\begin{array}[\textwidth]{llr}
	\gamma(b) \leq_C \gamma(a) 					& \implies & \text{per la monotonia di $\alpha$} \\
	\alpha(\gamma(b)) \leq_A \alpha(\gamma(a))  & \implies & \text{per l'ipotesi (B)} \\
	b \leq_A a									& \\
	\end{array}
	\end{displaymath}
	Siccome valgono al contempo $a \leq_A b$ e $b \leq_A a$ allora per l'antisimmetria del poset $(A, \leq_A)$
	$a = b$. Che è in contrasto con l'ipotesi iniziale che $a \neq b$.
	
	%% INIZIO DIMOSTRAZIONE (A) => (C) %
	$\boxed{(A) \implies (C)}$
	Assumo $\gamma$ iniettiva, ovvero $\forall a,b \in A . \ \gamma(a) = \gamma(b) \implies a = b$.
	Devo dimostrare che $\alpha$ suriettiva, ovvero:
	$$
	\forall a \in A .\ \exists c \in C .\ \alpha(c) = a
	$$  
	Per fare ciò assumo che $\alpha$ sia non suriettiva ovvero
	$$
	\exists a \in A .\ \forall c \in C .\ \alpha(c) \neq a
	$$
	e sia $a_0$ un elemento tale che $\alpha(c) \neq a_0 \forall c \in C$.
	$$
	\begin{array}{ll}
	\alpha(\gamma(a_0)) \leq_A a_0 & \text{Siccome è una GC} \\
	\gamma(\alpha(\gamma(a_0))) \leq_C \gamma(a_0) & \text{Poichè $\gamma$ monotona} \\	
	\end{array}
	$$
	Inoltre siccome $\gamma(a_0) \in C$ vale 
	$$
	\gamma(a_0) \leq \gamma(\alpha(\gamma(a_0)))
	$$
	Quindi unendo i due risultati (per l'antisimmetria di $\leq_C$) ottengo:
	$$
	\begin{array}{ll}
	\gamma(a_0) = \gamma(\alpha(\gamma(a_0))) & \text{Siccome $\gamma$ iniettiva} \\
	a_0 = \alpha(\gamma(a_0)) & \text{Contraddizione}
	\end{array}
	$$
	Avendo ottenuto una contraddizione ottengo che sotto l'assunzione $(A)$ non può esistere un $a_0 \in A$ tale che $\alpha(c) \neq a$ e quindi $\alpha$ è necessariamente suriettiva.
	
	
	%% INIZIO (C) => (B)
	$\boxed{(C) \implies (B)}$
	Assumo $\alpha$ suriettiva ovvero:
	$$
	\forall a \in A . \ \exists c \in C . \ \alpha(c) = a
	$$
	Assumo che non valga (B) $\alpha \circ \gamma \neq id$ ovvero:
	$$
	\exists a \in A : \alpha(\gamma(a)) \neq a
	$$
	Sia $a_0$ tale che $\alpha(\gamma(a_0)) \neq a_0$.
	Siccome $(\alpha, A, C, \gamma)$ è una $GC$ vale $\alpha(\gamma(a_0)) \leq a_0$.
	Dimostrando $a_0 \leq_A \alpha(\gamma(a_0))$ otterrei una contraddizione.
	
	Siccome abbiamo a che fare con una $GC$ (e $\alpha, \gamma$ totali):
	$$
	\gamma(\alpha(a_0)) \leq_C \gamma(\alpha(\gamma(a_0))) 
	$$
	Se $\gamma$ fosse iniettiva seguirebbe immediatamente il risultato cercato, ovvero
	$a_0 \leq (\alpha(\gamma(a_0)))$.
	Dimostro che $\gamma$ debba per forza essere iniettiva:
	\section{TODO: DIMOSTRA INIETTIVITÀ DI $\gamma$}
}
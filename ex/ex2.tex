\exercise{Esercizio 2}
{Let $C$ and $A$ be complete lattices and let $(\alpha, C, A, \gamma)$ be a Galois connection. Prove the following properties:
\begin{enumerate}
	\item $\gamma(\alpha(\top_C)) = \top_C$
	\item for any $a \in A, \gamma(a) = \wedge \{c \in C\ |\ \alpha(c) \leq_A a \}$ 
	\item for any $c_1, c_2 \in C, \alpha(c_1 \vee_C c_2) = \alpha(c_1) \vee_A \alpha(c_2)$
	\item for any $c \in C, \gamma(\alpha(\gamma(\alpha(c)))) = \gamma(\alpha(c))$
\end{enumerate}	
}
{
\begin{enumerate}
	\item $\boxed{\gamma(\alpha(\top_C)) = \top_C}$
	\begin{proof}
	Posso dimostrare le seguenti diseguaglianze per ottenere il risultato cercato:
	\begin{enumerate}
		\item
		\label{ex2:item:gamma:alpha:topC}
		$\gamma(\alpha(\top_C)) \leq_C \top_C$: Segue direttamente dal fatto che $\gamma:A \to C$.
		Infatti $\gamma$ restituisce un valore in $C$ che è sicuramente minore o uguale a $\top_C$.
		Il tutto vale perchè $\alpha$ e $\gamma$ sono funzioni totali.
		
		\item
		\label{ex2:item:topC}
		 $\top_C \leq_C \gamma(\alpha(\top_C))$: Segue direttamente dalla proprietà numero (3)
		($\forall c \in C . c \leq_C \gamma(\alpha(c))$) della Galois Connection.
	\end{enumerate}
	visto che valgono le diseguaglianze \ref{ex2:item:gamma:alpha:topC} e \ref{ex2:item:topC}, per l'antisimmetria del poset $(C, \leq_C)$ posso concludere
	$\gamma(\alpha(\top_C)) = \top_C$.
	\end{proof}
	\item $\boxed{\forall a \in A, \gamma(a) = \wedge \{c \in C\ |\ \alpha(c) \leq_A a \}}$
	\item $\boxed{\forall c_1, c_2 \in C, \alpha(c_1 \vee_C c_2) = \alpha(c_1) \vee_A \alpha(c_2)}$
	\item $\boxed{\forall c \in C, \gamma(\alpha(\gamma(\alpha(c)))) = \gamma(\alpha(c))}$
	Dimostro le due diseguaglianze separatamente:
	\begin{enumerate}
		\item $\boxed{\forall c \in C .  \gamma(\alpha(c)) \leq_C \gamma(\alpha(\gamma(\alpha(c))))}$
		segue direttamente dalla proprietà (4) e dalla totatilità di $\alpha$ e $\gamma$.
		\item $\boxed{\forall c \in C . \gamma(\alpha(\gamma(\alpha(c)))) \leq_C \gamma(\alpha(c))}$
		Per la proprietà (5) della Galois Connection (e la totalità di $\alpha$ e $\gamma$)vale la seguente relazione
		$$
		\alpha(\gamma(\alpha(c))) \leq_A \alpha(c) 
		$$
		vale per ogni $c$ perchè ogni $c$ viene mappato ad un $a$.
		Per via della monotonia di $\gamma$ ottengo:
		$$
		\gamma(\alpha(\gamma(\alpha(c)))) \leq_C \gamma(\alpha(c))
		$$
		che era la relazione da dimostrare.
	\end{enumerate}
	Ora siccome valgono le due relazioni per l'antisimmetria vale anche l'uguaglianza.
\end{enumerate}
	
}
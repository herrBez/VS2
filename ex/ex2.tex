\exercise{Esercizio 2}
{Let $C$ and $A$ be complete lattices and let $(\alpha, C, A, \gamma)$ be a Galois connection. Prove the following properties:
\begin{enumerate}
	\item $\gamma(\alpha(\top_C)) = \top_C$
	\item for any $a \in A, \gamma(a) = \vee_C \{c \in C\ |\ \alpha(c) \leq_A a \}$ 
	\item for any $c_1, c_2 \in C, \alpha(c_1 \vee_C c_2) = \alpha(c_1) \vee_A \alpha(c_2)$
	\item for any $c \in C, \gamma(\alpha(\gamma(\alpha(c)))) = \gamma(\alpha(c))$
\end{enumerate}	
}
{
\begin{enumerate}
	\item $\boxed{\gamma(\alpha(\top_C)) = \top_C}$
	\begin{proof}
	Posso dimostrare le seguenti diseguaglianze per ottenere il risultato cercato:
	\begin{enumerate}
		\item
		\label{ex2:item:gamma:alpha:topC}
		$\gamma(\alpha(\top_C)) \leq_C \top_C$: Segue direttamente dal fatto che $\gamma:A \to C$.
		Infatti $\gamma$ restituisce un valore in $C$ che è sicuramente minore o uguale a $\top_C$.
		Il tutto vale perchè $\alpha$ e $\gamma$ sono funzioni totali.
		
		\item
		\label{ex2:item:topC}
		 $\top_C \leq_C \gamma(\alpha(\top_C))$: Segue direttamente dalla proprietà numero (3)
		($\forall c \in C . c \leq_C \gamma(\alpha(c))$) della Galois Connection.
	\end{enumerate}
	visto che valgono le diseguaglianze \ref{ex2:item:gamma:alpha:topC} e \ref{ex2:item:topC}, per l'antisimmetria del poset $(C, \leq_C)$ posso concludere
	$\gamma(\alpha(\top_C)) = \top_C$.
	\end{proof}
	\item $\boxed{\forall a \in A, \gamma(a) = \vee_C \{c \in C\ |\ \alpha(c) \leq_A a \}}$
	
	\item $\boxed{\forall c_1, c_2 \in C, \alpha(c_1 \vee_C c_2) = \alpha(c_1) \vee_A \alpha(c_2)}$
	Siccome $C$ è un reticolo completo il sottoinsieme $\{c_1, c_2\}$ ha un lub che fa parte di $C$ (in ogni caso $\top_C$ è sempre un ub). Distinguo tre casi:
	\begin{enumerate}
		\item $c_1 \leq_C c_2$. Se vale questa relazione $c_1 \vee_C c_2 = c_2$ e quindi:
		\begin{equation}
		\label{ex2:eq:alpha(c1c2)}
		\alpha(c_1 \vee_C c_2) = \alpha(c_2)
		\end{equation}
		Inoltre
		$$
		\begin{array}{lll}
		c_1 \leq_C c_2 & \implies & \text{Monotonia di $\alpha$} \\
		\alpha(c_1) \leq_A \alpha(c_2) & &\\
		\end{array}
		$$
		Dunque il $lub(\{\alpha(c_1), \alpha(c_2)\}) = \alpha(c_2)$.
		Grazie a quest'ultimo risultato e a $\ref{ex2:eq:alpha(c1c2)}$ l'asserto è dimostrato.
		\item $c_2 \leq_C c_1$ riconducibile al precedente scambiando $c_1$ con $c_2$.
		\item $\neg (c_1 \leq_C c_2) \land \neg (c_2 \leq_C c_1)$. Sia $c$ il lub di $\{c_1, c_2\}$
		(Esiste perchè $C$ è un reticolo completo).
		Siccome $c$ è lub valgono le relazioni:
		\begin{itemize}
			\item $c_1 \leq_C c \implies \alpha(c_1) \leq_A \alpha(c)$ per la monotonicità di $\alpha$
			\item $c_2 \leq_C c \implies \alpha(c_2) \leq_A \alpha(c)$ per la monotonicità di $\alpha$.
		\end{itemize}
		Dimostro che $lub(\{\alpha(c_1), \alpha(c_2)\}) \leq lub(\{\alpha(c), \alpha(c)\}) = \alpha(c)$ che segue dalle due relazioni precedenti e dalla definizione di lub.
		
		\huge TODO dimostra $\alpha(c) \leq lub(\{\alpha(c_1), \alpha(c_2)\})$
	\end{enumerate}
	
	
	\item $\boxed{\forall c \in C, \gamma(\alpha(\gamma(\alpha(c)))) = \gamma(\alpha(c))}$
	Dimostro le due diseguaglianze separatamente:
	\begin{enumerate}
		\item $\boxed{\forall c \in C .  \gamma(\alpha(c)) \leq_C \gamma(\alpha(\gamma(\alpha(c))))}$
		segue direttamente dalla proprietà (4) e dalla totatilità di $\alpha$ e $\gamma$.
		\item $\boxed{\forall c \in C . \gamma(\alpha(\gamma(\alpha(c)))) \leq_C \gamma(\alpha(c))}$
		Per la proprietà (5) della Galois Connection (e la totalità di $\alpha$ e $\gamma$)vale la seguente relazione
		$$
		\alpha(\gamma(\alpha(c))) \leq_A \alpha(c) 
		$$
		vale per ogni $c$ perchè ogni $c$ viene mappato ad un $a$.
		Per via della monotonia di $\gamma$ ottengo:
		$$
		\gamma(\alpha(\gamma(\alpha(c)))) \leq_C \gamma(\alpha(c))
		$$
		che era la relazione da dimostrare.
	\end{enumerate}
	Ora siccome valgono le due relazioni, per l'antisimmetria di $\leq_C$ vale anche l'uguaglianza.
\end{enumerate}
	
}
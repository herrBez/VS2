\exercise{Esercizio 3}
{Let C and A be complete lattices, $(\alpha, C, A, \gamma)$ be a Galois insertion, $op : C^2 \to C$ be a monotone concrete operation and $op^a : A^2 \to A$ be a monotone abstract operation. Prove the following equivalence:
$$
\begin{array}{c}
	\forall (a_1, a_2) \in A^2\ .\ \alpha(op(\gamma(a_1), \gamma(a_2))) \leq_A op^a(a_1,a_2) \\
	\iff \\
	\forall (c_1, c_2) \in C^2\ .\ op(c_1, c_2) \leq_C \gamma(op^a(\alpha(c_1), \alpha(c_2)))
\end{array}
$$
}
{
\begin{mydef}(Galois Insertion)
Galois connection $+$ $\forall a \in A \alpha(\gamma(a)) = a$
\end{mydef}

Noto che $C^2$ e $A^2$ sono anch'essi dei reticoli completi:
\begin{itemize}
	\item $(C \times C, \leq_{C^2}, \sqcup_{C^2}, \bot_{C^2}, \top_{C^2})$
	\item $(A \times A, \leq_{A^2}, \sqcup_{A^2}, \bot_{A^2}, \top_{A^2})$
\end{itemize}
\begin{proof}
	Dimostro le due implicazioni separatamente.
\begin{itemize}
\item
$\boxed{\implies}$ 
Assumo che $\forall (a_1, a_2) \in A^2\ .  \alpha(op(\gamma(a_1), \gamma(a_2))) \leq_A op^a(a_1,a_2)$.
$$
\begin{array}{lr}
	op(c_1, c_2) \leq_C & \text{Perchè $op(c_1, c_2) \in C$ e abbiamo una Galois connection}\\
	\gamma(\alpha(op(c_1, c_2))) & \\
\end{array}
$$	
Inoltre valgono le seguenti relazioni
$$
\begin{array}{l}
c_1 \leq_C \gamma(\alpha(c_1))\\
c_2 \leq_C \gamma(\alpha(c_2))\\
\end{array}
$$
Siccome $op, \alpha, \gamma$ sono monotone
$$
\begin{array}{lr}
op(c_1, c_2) \leq_C & \text{Perchè $op(c_1, c_2) \in C$ e abbiamo una Galois connection}\\
\gamma(\alpha(op(c_1, c_2))) \leq_C & \text{Per le relazioni precedenti+$op, \alpha, \gamma$ sono monotone}\\
\gamma(\alpha(op(\gamma(\alpha(c_1)), \gamma(\alpha(c_2) )))) & \text{Per l'assunzione e la monotonicità di $\gamma$}\\
\gamma(op^A (\alpha(c_1), \alpha(c_2)))
\end{array}
$$

\item $\boxed{\impliedby}$
Assumo che $\forall (c_1, c_2) \in C^2\ .\ op(c_1, c_2) \leq_C \gamma(op^a(\alpha(c_1), \alpha(c_2)))$.

$$
\begin{array}{llr}
\alpha(op(\gamma(a_1), \gamma(a_2))) & \leq_A & \text{Assunzione + monotonia di $\alpha$}\\
\alpha(\gamma(op^a(\alpha(\gamma(a_1)),\ \alpha(\gamma(a_2))))) & = & \text{GI, i.e. $\alpha \circ \gamma = id$}\\
op^a(\alpha(\gamma(a_1)),\ \alpha(\gamma(a_2))) & = & \alpha \circ \gamma = id\\
op^a(a_1, a_2) & & \\
\end{array}
$$

\end{itemize}
\end{proof}


}